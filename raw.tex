\documentclass{article}
\usepackage[utf8]{inputenc}
\usepackage{lmodern}
\usepackage{geometry}
\geometry{a4paper, margin=1in}
\usepackage{hyperref}
\usepackage{booktabs}

\title{Pesquisa de Mercado --- Solução de IA para Monitoramento Jurídico em MG}
\author{OpenAI (Pesquisa simulada)}
\date{\today}

\begin{document}

\maketitle

\section{Introdução}
Este documento apresenta uma pesquisa de mercado voltada à viabilidade de uma solução de Inteligência Artificial (IA) para monitoramento de processos jurídicos no estado de Minas Gerais (MG). São compilados dados de fontes oficiais sobre:
\begin{itemize}
\item Número total de advogados ativos na OAB/MG;
\item Quantidade de escritórios de advocacia registrados na OAB/MG;
\item Número de novos advogados formados anualmente em MG;
\item Crescimento do número de advogados nos últimos anos;
\item Porte médio dos escritórios de advocacia (número de advogados/sócios);
\item Volume de processos judiciais em tramitação no estado (Tribunais Estadual, Federal e do Trabalho).
\end{itemize}

As referências utilizadas incluem o Conselho Federal da OAB, OAB/MG, CNJ, IBGE, MEC (INEP) e outras entidades oficiais, buscando dados atualizados (até 2025).

\section{1. Número de Advogados Ativos em Minas Gerais}
De acordo com o Conselho Federal da OAB, Minas Gerais é uma das seccionais mais populosas do Brasil em termos de advogados, atrás apenas de São Paulo. Os dados recentes indicam aproximadamente:
\begin{itemize}
\item \textbf{Advogados ativos em MG:} cerca de 143.870 profissionais inscritos regularmente na OAB/MG (Quadro da Advocacia, 2024/25), excluindo estagiários e inscrições suplementares.
\item Somando estagiários (cerca de 3.140) e suplementares (aprox. 6.191), chega-se a um total de mais de 153 mil inscritos em Minas Gerais.
\end{itemize}

Esse contingente representa cerca de 9,5\% do total de advogados do Brasil, que ultrapassa 1,42 milhão de inscritos ativos.

\section{2. Escritórios de Advocacia Registrados na OAB/MG}
A OAB também registra formalmente as sociedades de advogados. Em Minas Gerais, observa-se:
\begin{itemize}
\item \textbf{Sociedades de advogados ativas (2021):} 9.573 escritórios registrados, dos quais 4.235 são sociedades pluripessoais e 5.338 são sociedades individuais.
\item Em 2019, eram 7.095 sociedades, indicando um crescimento de cerca de 35\% em dois anos.
\end{itemize}

Observa-se ainda que aproximadamente 15,5\% dos advogados mineiros integram sociedades formalizadas. Muitos atuam como autônomos, em órgãos públicos ou em departamentos jurídicos de empresas privadas.

\section{3. Número de Novos Advogados Formados Anualmente em MG}
Minas Gerais abriga diversas faculdades de Direito, formando anualmente um grande número de bacharéis:
\begin{itemize}
\item Em 2019, havia cerca de 87,5 mil matrículas em cursos presenciais de Direito no estado, segundo dados do MEC/INEP.
\item Considerando um curso de 5 anos, estima-se que entre 15 mil e 18 mil alunos concluam a graduação anualmente em MG, embora haja evasão ao longo do período.
\end{itemize}

Nem todos os formados obtêm aprovação no Exame da OAB, mas o número de advogados no estado vem aumentando consistentemente, o que reflete o crescimento na formação de profissionais.

\section{4. Crescimento do Número de Advogados (2018--2024)}
O quadro de advogados em MG apresenta crescimento expressivo. A tabela a seguir ilustra dados de diferentes anos:

\begin{center}
\begin{tabular}{l l l}
\toprule
\textbf{Ano} & \textbf{Advogados Ativos em MG} & \textbf{Fonte} \\
\midrule
2017 & \(\sim 102.400\) & OAB/MG (Perfil 2017) \\
2019 & 115.646 & OAB/MG (Perfil 2019) \\
2021 & 132.001 & OAB/MG (Perfil 2021) \\
2024* & \(\sim 143.870\) & OAB -- Quadro Atualizado \\
\bottomrule
\end{tabular}
\end{center}

\emph{*2024 refere-se ao dado mais recente do Quadro da Advocacia atualizado em 2025.}

Pode-se observar um crescimento de aproximadamente 40\% no número de advogados em sete anos (2017--2024). Em termos percentuais, isso representa uma taxa anual de cerca de 5--7\%. Esse aumento contínuo, mesmo em período de adversidades econômicas, confirma a expansão do mercado jurídico local.

\section{5. Porte Médio dos Escritórios de Advocacia}
O perfil dos escritórios em MG é majoritariamente de pequeno porte:
\begin{itemize}
\item Há aproximadamente 9.573 sociedades de advocacia, reunindo cerca de 20.520 advogados.
\item Isso indica uma média de aproximadamente 2,1 advogados por escritório.
\item Mais da metade (55,7\%) dos escritórios é formada por sociedades individuais, com um único advogado titular.
\end{itemize}

As bancas pluripessoais também costumam ter poucos sócios, resultando em uma média de 3 a 4 advogados por sociedade nesses casos. Este dado é importante para a estratégia de precificação e funcionalidades de ferramentas de monitoramento processual, que devem ser escaláveis para atender desde profissionais autônomos até escritórios maiores.

\section{6. Volume de Processos Judiciais em Tramitação (MG)}
A demanda por monitoramento é impulsionada pelo alto volume de processos:
\begin{itemize}
\item De acordo com o Conselho Nacional de Justiça (CNJ), havia cerca de 83,8 milhões de processos pendentes em todo o Judiciário brasileiro em 2023, em todos os ramos.
\item A Justiça Estadual concentra mais de 80\% dos casos pendentes, o que inclui o Tribunal de Justiça de MG (TJMG).
\item O TJMG recebe entre 1,8 e 2 milhões de novos casos por ano, elevando seu estoque de ações a vários milhões em tramitação.
\item Além disso, Minas Gerais possui a Justiça do Trabalho (TRT 3ª Região) e o recém-criado Tribunal Regional Federal da 6ª Região (TRF-6), que adicionam centenas de milhares de ações ao volume processual total no estado.
\end{itemize}

Em síntese, estima-se que haja alguns milhões de processos em tramitação simultânea em MG, distribuídos entre Justiça Estadual, Federal e do Trabalho. Esse cenário de elevado contencioso confirma a oportunidade de soluções automatizadas de monitoramento.

\section{Considerações Finais}
Os dados apontam que Minas Gerais:
\begin{enumerate}
\item Possui mais de 140 mil advogados ativos, em forte crescimento;
\item Conta com milhares de escritórios, na maioria de pequeno porte;
\item Forma continuamente milhares de novos bacharéis (e subsequentes advogados) por ano;
\item Gera milhões de processos em tramitação, exigindo acompanhamento permanente.
\end{enumerate}

Esses fatores favorecem a adoção de uma solução de inteligência artificial para acompanhar e gerenciar o grande volume de ações, prazos e movimentações. Com a alta densidade de advogados e processos, a demanda por automatização tende a ser significativa, especialmente para escritórios pequenos e médios, que não dispõem de departamentos internos de tecnologia robustos.

\section*{Referências Consultadas}
\begin{itemize}
\item Conselho Federal da OAB. \emph{Quadro da Advocacia} (dados atualizados até 2025).
\item OAB/MG. Perfis da Advocacia Mineira (2017, 2019, 2021).
\item CNJ. \emph{Justiça em Números 2024/2025}.
\item MEC/INEP. \emph{Censo da Educação Superior} (2018--2022).
\item IBGE. \emph{Estimativas da População} (2022).
\end{itemize}

\end{document}
